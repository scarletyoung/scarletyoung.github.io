\documentclass[a4paper]{article}
\title{数值分析笔记}
\author{scarlet young}
\date{\today}

\usepackage[utf8]{inputenc}
\usepackage[T1]{fontenc}
\usepackage{textcomp}
\usepackage[dutch]{babel}
\usepackage{amsmath, amssymb}
\usepackage{ctex}


% figure support
\usepackage{import}
\usepackage{xifthen}
\pdfminorversion=7
\usepackage{pdfpages}
\usepackage{transparent}
\newcommand{\incfig}[1]{%
	\def\svgwidth{\columnwidth}
	\import{./figures/}{#1.pdf_tex}
}

\pdfsuppresswarningpagegroup=1

\begin{document}
\section{插值法}
\subsection{引言}
\newtheorem{definition}{定义}
\begin{definition}
	设函数$f(x)$在$[a,b]$上有定义,且已知$a \le x_0 < x_1 \cdots <x_{n} \le b $点上的值$y_0,y_1,\cdots, y_{n} $,若存在以简单函数$\phi(x)$,使得
	\[
		\phi(x_{i}) = y_{i} \ \ \ \ i=0,1,2,\cdots n \tag{2.1} \label{2.1}
	.\] 
	成立,则称$\phi(x)$ 为$f(x)$ 的插值函数。式$\ref{2.1}$称为插值条件, $f(x)$ 称为被插值函数,$[a,b]$ 称为插值区间,$x_0,x_1,\cdots,x_{n}$ 称为插值节点,求$\phi(x)$ 的方法就是插值法。
\end{definition}
研究问题
\begin{enumerate}
	\item $\phi(x)$ 是否存在,是否唯一
	\item 若存在,如何构造$\phi(x)$	
	\item 如何估计$\phi(x)$的误差
\end{enumerate}
\subsection{Lagrange插值多项式}
当$n=1$ 时,要构造通过两点$(x_0,y_0)$ 和$(x_1,y_1)$ 的不超过一次的多项式$L_1(x)$,使得
\[
	\begin{cases}
		L_1(x_0) = y_0 \\
		L_1(x_1) = y_1
	\end{cases}
.\]
$L_1(x)$ 的表达式为
\[
L_1(x) = \frac{x_1-x}{x_1-x_0} y_0 + \frac{x - x_0}{x_1 - x_0}y_1
.\] 
$L_1(x)$ 表达式可以看作是函数值$y_0$ 和$y_1$ 的线性组合,组合的系数记为$l_0(x)$ 和$l_1(x)$,即
\[
\begin{cases}
	l_0 = \frac{x - x_1}{x_0 - x_1} \\
	l_1 = \frac{x - x_0}{x_1 - x_0}
\end{cases}
.\] 
系数$l_0(x)$和$l_1(x)$不是常数,而是一次多项式,因此,组合后的结果也是一次多项式。$l_0(x)$和$l_1(x)$称为节点$x_0,x_1$ 上的线性插值基函数。线性插值基函数还需要满足插值条件(见表\ref{tab:label})。
\begin{table}[htbp]
	\centering
	\caption{interpolate condition}
	\label{tab:label}
	\begin{tabular}{|c|c|c|}
		\hline
		& $x_0$ & $x_1$ \\
		\hline
		$l_0(x)$ & 0 & 1 \\
		\hline
		$l_1(x)$ & 1 & 0 \\
		\hline
	\end{tabular}
\end{table}

根据基函数的表达式可以得知,基函数的构造与函数值无关。

将上述公式推广到一般情形。
通过$n+1$ 个节点的n次插值多项式$L_n(x)$,设$L_n(n) = y_0l_0(x) + y_1l_1(x) + \cdots + y_{n}l_{x}(n)$满足插值条件$L_n(x_j) = y_j, \ \ j = 0,1,\cdots,n$。
\newtheorem{definition}{定义}
\begin{definition}
若$n$ 次多项式$l_k(x)(k=0,1,\cdots,n$在各节点$x_0 < x_1 < \cdots < x_n$ 上满足条件,
\[
	l_k(x_{i}) = \delta_{ki} = \begin{cases}
		1 & k = i \\
		0 & k \neq i
	\end{cases} \ \ i,k = 0,1,\cdots,n \tag{2.2} \label{2.2}
.\] 
就称这$n+1$ 个$n$ 次多项式$l_0(x), l_1(x), \cdots, l_n(x)$ 为节点$x_0, x_1, \cdots, x_{n}$ 上的n次插值基函数。
\end{definition}
用类推的方式可以得到$n$ 次插值基函数为
\[
	l_k(x) = \prod_{k=1, k \neq j}^{n} \frac{x - x_{j}}{x_{k} - x_{j}} 
.\] 
于是,插值多项式函数可以表示为
\[
	L_n(x) = \sum_{k=0}^{n} y_k l_k(x) \tag{2.3} \label{2.3}
.\] 
形如式$\ref{2.3}$的插值多项式$L_n(x)$ 称为Lagrange插值多项式。
引入记号
\[
	\omega_{n+1}(x) = (x - x_0)(x - x_1)\cdots(x - x_{n})
.\] 
可知
\[
	\omega_{n+1}'(x) = (x_k - x_0) \cdots (x_k - x_{k-1})(x_k - x_{k+1}) \cdots (x_k - x_{n})
.\] 
于是式$\ref{2.3}$ 可以改写为
\[
	L_n(x) = \sum_{k=0}^{n} y_k \frac{\omega_{n+1}(x)}{(x-x_k)\omega_{n+1}'(x_k)}
.\]

若在$[a,b]$ 上用$L_n(x)$ 近似$f(x)$ ,则其截断误差为$R_n(x) = f(x) - L_n(x)$,$R_n(x)$ 也称为插值多项式的余项或插值余项。
\newtheorem{theorem}{定理}
\begin{theorem}
	设$f^{(n)}(x)$ 在$[a,b]$ 上连续,$f^{(n+1)}(x)$ 在$(a,b)$ 内存在,节点$a \le x_0 < x_1 < \cdots < x_n \le b $,$L_n(x)$是满足插值条件的插值多项式,则对于任何$x \in [a,b]$,插值余项为
	 \[
		 R_n(x) = f(x) - L_n(x) = \frac{f^{(n+1)}(\xi)}{(n+1)!}\omega_{n+1}(x) \tag{2.4} \label{2.4}
	.\] 
	其中,$\xi \in (a,b)$且依赖与$x$ 。
\end{theorem}

\subsection{逐次线性插值法}

\subsection{Newton插值多项式}

\subsection{Hermite插值多项式}

\subsection{分段低次插值}

\subsection{三次样条插值}
	

\end{document}

