\documentclass[a4paper]{article}
\title{动态规划}
\author{scarlet young}
\date{\today}

\usepackage[utf8]{inputenc}
\usepackage[T1]{fontenc}
\usepackage{textcomp}
\usepackage[dutch]{babel}
\usepackage{amsmath, amssymb}
\usepackage{listings}
\usepackage{xcolor}
\usepackage{ctex}

% figure support
\usepackage{import}
\usepackage{xifthen}
\pdfminorversion=7
\usepackage{pdfpages}
\usepackage{transparent}
\newcommand{\incfig}[1]{%
	\def\svgwidth{\columnwidth}
	\import{./figures/}{#1.pdf_tex}
}

\pdfsuppresswarningpagegroup=1
\lstset{
    basicstyle          =   \sffamily,
    keywordstyle        =   \bfseries,
    commentstyle        =   \rmfamily\itshape,
    stringstyle         =   \ttfamily,
    flexiblecolumns,               
    numbers             =   left,
    showspaces          =   false,
    numberstyle         =   \zihao{-5}\ttfamily,
    showstringspaces    =   false,
    captionpos          =   t,
    frame               =   lrtb,
}
\begin{document}
\section{introduction}
动态规划是一种编程范式,可以系统地、有效地探索问题的所有可能解决方案。动态规划问题通常具有以下特征
\begin{enumerate}
	\item 问题可以划分为可重叠的子问题,子问题的解可以被多次重用
	\item 问题有最优子结构,即问题的最优解可以有子问题的最优解得出
\end{enumerate}
动态规划(dp)的一个简单例子是斐波那契数列。
对于第$n^{th}$斐波那契数$F(n)$ ,可以分解为两个小的子问题$f(n-1)$ 和$f(n-2)$ 。
问题具有最优子结构,即$f(n) = f(n-1) + f(n-2)$。
子问题是重叠的,即对于$f(5)$ 和$f(6)$ 都需要计算$F(4)$。

贪心问题与dp问题的区别在于,贪心问题有最优子结构,但没有重叠的子问题。
分治算法与dp问题的区别在于,分支算法可以将问题分解为子问题,但子问题是不重叠的。

实现dp算法有两种方式,自下而上和自上而下
自下而上的实现从最基本的例子开始迭代实现。
\begin{lstlisting}
F = array of length n+1
F[0] = 0
F[1] = 1
for i from 2 to n:
	F[i] = F[i-1] + F[i-2]
\end{lstlisting}

自上而下使用递归实现,同时,为了避免重复计算,缓存了中间结果。
\begin{lstlisting}
memo = hashmap
Function F(integer i):
	if i is 0 or 1:
		return i
	if i doesn't exist in memo:
		memo[i] = F(i-1) + F(i-2)
	return memo[i]	
\end{lstlisting}

自下而上的方式通常更快,自上而下的实现更加容易编写,而且不需要考虑子问题的顺序。

\end{document}

